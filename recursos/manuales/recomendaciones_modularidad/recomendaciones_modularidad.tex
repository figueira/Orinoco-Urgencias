\documentclass[9pt, letterpaper, oneside]{report}
\usepackage[utf8]{inputenc}
\usepackage[spanish]{babel}
\usepackage{hyperref}
\usepackage[margin=2cm]{geometry}

\hypersetup{
  pdfstartpage = 1, 
  pdftitle = {Recomendaciones para obtener mayor Modularidad},
  pdfauthor = {Jonathan Queipo},%
  pdfcreator = {Jonathan Queipo},
  pdfproducer = {LaTeX with hyperref},
  pdfdisplaydoctitle = true,
  pdflang = es,
  plainpages = false,
}

% Definicion del titulo
\title{Recomendaciones para obtener mayor Modularidad}
\author{Jonathan Queipo}
\date{\today}

\begin{document}

\maketitle

\newpage

\tableofcontents

\newpage

\chapter{Introducción}

  En el presente documento se plantean algunas sugerencias y recomendaciones para obtener
  mayor modularidad en el programa TDS. Esto debido a la creciente necesidad de
  descentralizar la programación, que actualmente se encuentra notoriamente concentrando
  en una sola aplicación de todo el programa, la cual es \textbf{app\_emergencia}.
  
  Para mayor escalabilidad y mantenibilidad del desarrollo realizada, es de importancia
  crítica poder modularizar el código de la aplicación app\_emergencia, y distribuir el
  mismo a través de las aplicaciones ya existentes o, de ser necesario, en aplicaciones
  nuevas.
  
  Durante el transcurso del documento nos referiremos a el programa TDS como 
  la \textbf{aplicación} y cada componente de la misma como \textbf{módulos}.
  
  Dicho todo esto, procedemos a exponer las recomendaciones.
  

\chapter{Recomendaciones}

  \section{Interfaz Gráfica}
    Actualmente la página de inicio de la aplicación, lugo de haber realizado
    inicio de sesión, se presenta accesos a las
    cuatro (4) áreas principales de TDS, \textbf{Ingreso}, \textbf{Listados},
    \textbf{Historia Médica} y \textbf{Estadísticas}. En el presente estado de
    la apliación el módulo de Historia Médica (HM) se encuentra inhabilitado, sin
    embargo sigue apareciendo en la página de inicio, y al dar click sobre el
    logo que le representa nos dirige hacia el listado de pacientes atendidos.
    
    Lo ideal sería que el módulo HM no esté presente en la página de inicio, pero
    sin eliminar el código de la interfaz gráfica, para que se deje abierta la
    posibilidad de reincorporarlo a la vista en el momento en el que se termine
    de desarrollar el módulo.
    
    Se plantean dos opciones\footnote{Queda a total facultad del lector el
    aceptar o rechazar las sugerencias aquí expuestas, pues son simplemente eso...
    sugerencias. No obstante estas recomendaciones son el resultado de un trimestre
    de desarrollo y de ensayo y error en otras áreas de la aplicación, lo que nos
    brindó cierta experiencia.}:
    
    \subsection{Creación de nueva plantilla HTML}
    
    En principio la opción más evidente sería crear una plantilla nueva que no
    contenga la parte gráfica del módulo HM, diferente a \textbf{loged.html} que
    es la plantilla actual de la vista posterior al inicio de sesión.
    
    La idea es que al momento de desear habilitar o deshabilitar el módulo HM
    se puede \emph{jugar} con las redirecciones en los archivos \textbf{urls.py}.
    
    \subsection{Creación de tabla auxiliar en la base de datos}

  	La segunda opción que se plantea es la creación de una tabla auxiliar en la
  	base de datos que almacene los nombres, nombres de imágenes, urls bases y demás
  	información relevante sobre las aplicaciones instaladas.
  	
  	Cuando ya se tiene la información almacenada en la base de datos, se puede hacer
  	una consulta desde el \textbf{views.py} que nos devuelve todas las aplicaciones
  	instaladas al momento de utilizar el sistema, sin la necesidad de crear más de
  	una plantilla para la página sucesiva al inicio de sesión exitoso.
  	
  	

\end{document}
