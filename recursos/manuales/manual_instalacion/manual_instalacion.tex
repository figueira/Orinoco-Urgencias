\documentclass[9pt, letterpaper, oneside]{report}
\usepackage[utf8]{inputenc}
\usepackage[spanish]{babel}
\usepackage{hyperref}
\usepackage[margin=2cm]{geometry}

\hypersetup{
  pdfstartpage = 1, 
  pdftitle = {Manual de instalación TDS},
  pdfauthor = {Carlos Ledezma},%
  pdfcreator = {Carlos Ledezma},
  pdfproducer = {LaTeX with hyperref},
  pdfdisplaydoctitle = true,
  pdflang = es,
  plainpages = false,
}

% Definicion del titulo
\title{Manual de instalación del sistema TDS}
\author{Carlos Ledezma}
\date{\today}

\begin{document}

\maketitle

\newpage

\tableofcontents

\newpage

\chapter{Introducción}

  A continuación se presenta un documento con las instrucciones detalladas de cómo
  instalar la aplicación TDS en un sistema operativo Windows 8, recién instalado,
  de 32 bits, configurado en inglés. Es recomendable que durante todo el proceso
  se cuente con una conexión estable a internet.

  Si se está haciendo la instalación en un sistema con características diferentes,
  algunos pasos deben ser adaptados a dichas características. En particular es muy
  común el cambio de un sistema de 32-bits a uno de 64-bits. En dicho caso
  particular sólo es necesario hacer la salvedad de descargar los programas en la
  versión apropiada.

  A lo largo del proceso se instalarán diversos programas, es conveniente que se
  verifique la correcta instalación de los mismos antes de seguir con pasos 
  subsiguientes.

  Así pues, empecemos.

\chapter{Dependencias}

  \section{GitHub}
    Dado que el sistema se encuentra alojado en un repositorio de GitHub, será
    conveniente descargar la aplicación que dicha página nos brinda para manipular
    los repositorios. Para ello vaya a la siguiente dirección:

    \vspace{3mm}
    \url{http://windows.github.com/}
    \vspace{3mm}

    y oprima el botón que dice \textbf{Download GitHub for Windows}.

    Cuando finalice la descarga, inicie el instalador haciendo doble click en el
    archivo obtenido. En el dialogo inicial haga click en \textbf{Install} y espere a que
    finalice el proceso de descarga.

    Inicie el instalador, coloque sus credenciales de GitHub y realice los pasos de
    configuración. Es posible que la aplicación tenga problemas recuperando las
    credenciales del servidor de GitHub. Si ésto ocurre, pruebe descargando la
    última actualización para el ``.NET Framework'', que se puede conseguir en:

    \vspace{3mm}
    \url{http://support.microsoft.com/kb/2468871}
    \vspace{3mm}

  \section{Python}
    El programa se ejecuta utilizando el framework ``Django'', escrito en el lenguaje de
    programación ``Python'', por lo que será necesario descargar e instalar el
    intérprete para este lenguaje. El mismo se puede conseguir en:

    \vspace{3mm}
    \url{http://www.python.org/getit/windows/}
    \vspace{3mm}
      
    en dicha página haga click en el enlace que dice \textbf{individual
    releases}. En la página
    siguiente haga click en el enlace \textbf{Python 2.7.6}. Busque en la página siguiente
    el enlace \textbf{Windows x86 MSI Installer (2.7.6) (sig)} y haga click en él.

    Puede que no se encuentre la versión específica, en dicho caso es importante
    descargar la versión más cercana, asegurándose que la misma se mantiene en el
    rango 2.7.x. Ésto se debe a que en la versión 3.x se hicieron importantes
    cambios sobre el lenguaje, los cuales no han sido contemplados en el código.

    En la primera pantalla seleccione \textbf{Install for all users} y haga
    click en \textbf{Next}.
    Luego tome nota del directorio de instalación, al cual de ahora en adelante
    llamaremos $<python\_home>$, y haga click en \textbf{Next} hasta que inicie la
    instalación. Cuando finalice la instalación haga click en \textbf{Finish} y reinicie
    la computadora.

    Para utilizar correctamente Python será necesario cambiar las variables de
    ambiente de Windows. Para ésto vaya al \textbf{Control Panel}, busque la opción
    \textbf{Edit the system environment variables}. En la caja de diálogo que aparece haga
    click en \textbf{Environment Variables...}. En el panel ``System variables'' busque la
    variable ``Path'', seleccionela y haga click en \textbf{Edit...}. Al final del valor
    ``Variable value'' AGREGUE el valor:

    \begin{verbatim}
    ;<python_home>;<python_home>\Scripts
    \end{verbatim}

    donde $<python\_home>$ es el directorio en el que se instaló Python, y del que
    tomó nota en el paso de instalación. Cuando haya hecho ésto, haga click en
    \textbf{Ok}.

  \section{Pip}
    Para instalar ciertas extensiones de Python que son necesarias para la aplicación
    es conveniente utilizar ``Pip'', un manejador de paquetes de Python. Para ello tome
    el contenido de:

    \vspace{3mm}
    \url{https://bitbucket.org/pypa/setuptools/raw/bootstrap/ez\_setup.py}
    \vspace{3mm}
      
    y péguelo en un archivo en su escritorio con el nombre ``ez\_setup.py''.

    Haga doble click sobre el archivo generado. Ésto debería iniciar una consola de 
    Python que ejecutará una serie de comandos y que, al finalizar, se cerrará sola.

    Si al hacer doble click no se inicia la consola, entonces haga click derecho
    sobre el archivo, escoja la opción \textbf{Open With...} y escoja manualmente el
    intérprete de Python, el cual encontrará bajo el directorio:

    \begin{verbatim}
    <python_home>\python.exe
    \end{verbatim}

    Ahora copie el contenido de la página:

    \vspace{3mm}
    \url{https://raw.github.com/pypa/pip/master/contrib/get-pip.py}
    \vspace{3mm}
      
    y péguelo nuevamente en el escritorio, en un archivo llamado 'get-pip.py' y haga
    doble click sobre el archivo. Nuevamente se iniciará la consola de Python. Si no
    inicia, seguir los mismos pasos que en la sugerencia anterior.

  \section{Django}
    Ahora que se tiene Pip, se puede fácilmente instalar el framework Django 
    abriendo la aplicación ``PowerShell'' con derechos de administrador y colocando el
    siguiente comando:

    \begin{verbatim}
    pip install Django
    \end{verbatim}

    Se mostrará el progreso de la descarga. Espere a que finalice.

    Para verificar que Django se instaló correctamente ejecute el intérprete de
    Python corriendo en el ``PowerShell'' el siguiente comando:

    \begin{verbatim}
      python
    \end{verbatim}

    y en la consola que obtendrá coloque:

    \begin{verbatim}
      import django
      django.VERSION
    \end{verbatim}
      
    Lo que debería dar un resultado parecido a ``(1, 6, 0, 'final', 0)'',
    que indica
    la version del framework instalada. Puede salir de la consola interactiva de
    Python con el comando:

    \begin{verbatim}
      quit()
    \end{verbatim}

  \section{PostgreSQL}
    Luego será necesario descargar la base de datos. El programa a utilizar será
    PostgreSQL, el cual se puede descargar de:

    \vspace{3mm}
    \url{http://www.enterprisedb.com/products-services-training/pgdownload#windows}
    \vspace{3mm}
      
    haciendo click en el link ``Win x86-32'' bajo la versión 9.3.1. Cuando finalice la
    descarga, arranque el instalador. Acepte las configuraciones por defecto, y
    escoja una clave para el super-usuario cuando se le pida la misma. Tome nota de
    la clave que escogió y espere a que finalice la instalación.

    Desmarque la opción para lanzar el ``Stack Builder'' y haga click en
    \textbf{Finish}.

  \section{Compilador de C}
    Más adelante será necesario instalar algunas extensiones de Django que la aplicación
    utiliza. Para ello será necesario tener, primero, un compilador de C. Éste
    se puede obtener en:

    \vspace{3mm}
    \url{http://tdm-gcc.tdragon.net/download}
    \vspace{3mm}
      
    en el enlace que indica la versión de 32-bits.

    Cuando finalice la descarga ejecute el instalador. En la primera pantalla
    seleccione la opción \textbf{Create}. Cuando finalice la descarga seleccione la opción
    de 32-bits y haga click en \textbf{Next}. En adelante acepte las opciones por defecto 
    hasta que inicie la instalación. Espere a que la misma finalice.

    Ahora cree el archivo
    
    \begin{verbatim}
      <python_home>\Lib\distutils\distutils.cfg'
    \end{verbatim}
    
    con el siguiente contenido:

    \begin{verbatim}
      [build]
      compiler=mingw32
    \end{verbatim}

  \section{Extensiones de Django}
    Para utilizar la base de datos será necesario instalar un paquete de Python que 
    hace la interfaz entre el lenguaje y la base de datos. Para ello vaya a la
    siquiente página:
      
    \vspace{3mm}
    \url{http://www.stickpeople.com/projects/python/win-psycopg/}
    \vspace{3mm}
      
    y descargue el archivo en la fila '2.5.1 (For Python 2.7)'. Al finalizar la
    descarga ejecute el instalador usando los valores que se dan por defecto.

    También será necesario obtener algunas extensiones de Django. Para ello abra
    un ``PowerShell'' y escriba los comandos:

    \begin{verbatim}
      pip install django_extensions
      pip install Pillow
      pip install pisa
      pip install reportlab
      pip install html5lib
    \end{verbatim}

    Si se genera algún error durante la instalación de alguna de estas extensiones,
    lo que reconocerá por un mensaje de error, resaltado e rojo, que aparecerá en la
    consola, pruebe cerrar PowerShell, volver a abrirlo y reintentar la instalación.
    Si ésto falla, deberá descargar manualmente el complemento que falla e 
    instalarlo con
    los programas que se brindan. Los mismos se pueden encontrar en:

    \begin{description}
      \item[django\_extensions:]
        \url{https://github.com/django-extensions/django-extensions}
      \item[Pillow:] \url{https://pypi.python.org/pypi/Pillow/2.3.0#downloads}
      \item[pisa:] \url{https://pypi.python.org/pypi/pisa/#downloads}
      \item[reportlab:]
        \url{http://www.reportlab.com/software/opensource/rl-toolkit/download/}
      \item[html5lib:] \url{https://pypi.python.org/pypi/html5lib}
    \end{description}
      
    En cualquiera de estas páginas buscar preferiblemente la versión del programa
    que indique que es un ``Windows Installer'' o similar y descargar la versión más
    actualizada, adecuada al sistema de operación que se tenga. De no contar con un
    instalador para Windows, debe seguir las instrucciones que se den en la página
    particular para hacer una instalación manual del paquete.

\chapter{Configuración de la base de datos}

  Para realizar esta configuración inicie la aplicación ``pgAdmin III'', instalada
  con la base de datos. 

  En la lista de la izquierda haga doble click sobre el único servidor que
  debería haber (PostgreSQL 9.3 (x86) (localhost:5432)) e ingrese la contraseña
  escogida para el super-usuario de la base de datos.

  Debería obtener varias listas desplegables bajo el servidor. Haga click derecho
  sobre la lista ``Databases'' y seleccione la opción \textbf{New Database...}.
  Asigne a 'Name' el valor:
  
  \begin{verbatim}
    cmsb
  \end{verbatim}
  
  y en 'Owner' escoja de la lista desplegable el valor
  
  \begin{verbatim}
    postgres
  \end{verbatim}
  
  Haga click en \textbf{Ok}.

\chapter{Instalación de la aplicación}

  Para obtener el cógigo fuente de la aplicación, el mismo debe ser descargado
  desde el repositorio en GitHub al cual habrá sido invitado como colaborador. Si
  no ha sido invitado como colaborador, debe solicitar dicha autorización al
  equipo de desarrollo de la aplicación.

  Lance la aplicación de GitHub que se instaló en pasos anteriores. Bajo la pestaña
  ``github'' tendrá su nombre de usuario. Haga click en él y busque en la derecha el
  repositorio ``ramiroso/TDScopia''. Haga click en \textbf{clone}, la opción
  que aparece al pasar el mouse sobre el nombre del repositorio.

  Una vez que se termine de copiar el repositorio, haga click sobre la opción
  \textbf{repositories} bajo la pestaña ``local'' y haga click sobre la flecha que se
  encuentra a la derecha del nombre del repositorio que se acaba de clonar.

  En la parte superior derecha encontrará la imágen de un engranaje. Haga click
  sobre ella, y luego sobre la opción \textbf{open in explorer}. Ésto abrirá
  la ubicación en la computadora del código de la aplicación.

  Si se hace click sobre la barra superior (donde se encuentra la ubicación) se 
  mostrará la ruta real del archivo. A ésta la llamaremos $<home\_tds>$.

  Ahora será necesario realizar ciertos pasos de configuración sobre la aplicación.
  Para ello abra en un editor de texto (Notepad, Notepad ++) el archivo:

  \begin{verbatim}
    <home_tds>/AM/settings.py
  \end{verbatim}

  Vaya a la opción llamada ``DATABASES'' y asigne a la entrada ``ENGINE'' el
  valor:

  \begin{verbatim}
    'django.db.backends.postgresql_psycopg2'
  \end{verbatim}
  
  borrando la opción que se encuentre presente. A la opción ``NAME'' asígnele el
  valor: 
  
  \begin{verbatim}
    'cmsb'
  \end{verbatim}

  A la opción ``USER'' asignele el valor:
  
  \begin{verbatim}
    'postgres'
  \end{verbatim}

  y a ``PASSWORD'' la contraseña escogida durante la instalación de la base de
  datos, escrita entre comillas. Guarde los cambios.

  Ahora abra ``PowerShell'' y escriba los siguientes comandos:

  \begin{verbatim}
    cd <home_tds>
    python manage.py syncdb
  \end{verbatim}
    
  Cuando se pregunte si se desea crear un super-usuario para la aplicación escribir

  \begin{verbatim}
    yes
  \end{verbatim}

  y escoger un nombre de usuario, correo electrónico y contraseña.

  Cuando termine la sincronización de la base de datos puede ejecutar en la misma
  consola:

  \begin{verbatim}
    python manage.py runserver
  \end{verbatim}

  que iniciará la aplicación, a la que se podrá acceder con cualquier navegador
  (preferiblemente Mozilla Firefox) navegando a la url:

  \vspace{3mm}
  \url{localhost:8000}
  \vspace{3mm}

\chapter{Otras consideraciones}
  \section{Lista de instaladores}
    A continuación se presenta una lista de los programas que fueron descargados
    durante el proceso de instalación descrito anteriormente, junto con las
    direcciónes en las cuales se pueden conseguir. Podría ser útil visitar cada una
    de las páginas mencionadas, antes de iniciar el proceso, de modo que se
    puedan descargar en paralelo todos los programas, y así se finalice más
    rápido.

    \begin{description}
      \item[GitHub:] \url{http://windows.github.com/}

      \item[Actualización para ``.NET Framework'':]
        \url{http://support.microsoft.com/kb/2468871}

      \item[Python:] \url{http://www.python.org/getit/windows/}

      \item[Pip:]
        \begin{itemize}
          \item \url{https://bitbucket.org/pypa/setuptools/raw/bootstrap/ez\_setup.py}
          \item \url{https://raw.github.com/pypa/pip/master/contrib/get-pip.py}
        \end{itemize}

      \item[PostgreSQL:]
        \url{http://www.enterprisedb.com/products-services-training/pgdownload#windows}

      \item[Compilador de C:] \url{http://tdm-gcc.tdragon.net/download}
      
      \item[Psycopg:] \url{http://www.stickpeople.com/projects/python/win-psycopg/}

      \item[django\_extensions:]
        \url{https://github.com/django-extensions/django-extensions}

      \item[Pillow:] \url{https://pypi.python.org/pypi/Pillow/2.3.0#downloads}

      \item[pisa:] \url{https://pypi.python.org/pypi/pisa/#downloads}

      \item[reportlab:]
        \url{http://www.reportlab.com/software/opensource/rl-toolkit/download/}

      \item[html5lib:] \url{https://pypi.python.org/pypi/html5lib}

    \end{description}

  \section{Dependencias de nombres}

    En su versión actual, el código de TDS depende de que ciertos elementos
    introducidos posean nombres particulares. Éste es el caso de las áreas de
    atención que se pueden visualizar.

    Actualmente el sistema permite observar una ``Sala de atención ambulatoria''
    y una ``Sala de observación''. Para garantizar el correcto funcionamiento de
    dichas vistas, al momento de agregar las áreas de emergencia al sistema se
    debe garantizar que haya una cuyo nombre incluya la palabara ``ambulatoria''
    y una que incluya la palabra ``observación''. Si ésto se configura
    correctamente, los pacientes que se asignen a cubículos en dichas áreas,
    serán correctamente mostrados en las pantallas correspondientes.

    Es un objetivo a corto plazo corregir ésto para garantizar independencia de
    la elección del nombre del área de atención. Cuando esta meta se logre, el
    hecho será notificado a las instancias apropiadas.

\end{document}
